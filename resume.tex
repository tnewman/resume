%!TEX TS-program = xelatex
%!TEX encoding = UTF-8 Unicode

% Configuration
\documentclass[11pt, a4paper]{awesome-cv}
\geometry{left=1.4cm, top=.8cm, right=1.4cm, bottom=1.8cm, footskip=.5cm}
\fontdir[./fonts/]
\definecolor{awesome}{HTML}{007185}
\setbool{acvSectionColorHighlight}{true}
\renewcommand{\acvHeaderSocialSep}{\quad\textbar\quad}

% Contact Information
\name{Thomas}{Newman}
\position{Software Engineer}
\address{REDACTED}

\mobile{REDACTED}
\email{REDACTED}
\github{tnewman}
\linkedin{thomasjnewman}

\begin{document}

\makecvheader[C]

\cvsection{Skills}

\begin{cvskills}
    \cvskill
    {DevOps}
    {AWS, Azure, Docker, Google Cloud, Gradle, Jenkins, Kubernetes, Linux, Terraform}

    \cvskill
    {Backend}
    {Spring Boot, Flask, Microservices, MongoDB, PubSub, REST}

    \cvskill
    {Programming}
    {C\#, Java, JavaScript, Python}
\end{cvskills}

\cvsection{Experience}
\begin{cventries}
    \cventry
    {Software Engineer II - Backend}
    {DENSO International America, Inc.}
    {Southfield, MI}
    {February 2018-Present}
    {
        \begin{cvitems}
            \item{Wrote microservices in Python (Flask) and Java (Spring Boot) for a car sharing platform.}
            \item{Worked with external partners to implemented REST and Publish/Subscribe-based APIs for a car sharing platform.}
            \item{Created the initial microservices-based architecture for a car sharing platform.}
            \item{Wrote a Python AsyncIO library for Azure Service Bus capable of sending and receiving 300 messages per second on a laptop, which was triple the throughput of the supplied library.}
            \item{Implemented a Continuous Integration (CI)/Continuous Deployment (CD) in Jenkins. Jenkins script libraries were created to allow developers to build and deploy applications packaged as Docker containers to Kubernetes with a single-line Jenkins file.}
            \item{Provisioned infrastructure as code using Terraform in AWS. The Terraform plans were deployed by Jenkins with environments configured based on Git branches, ensuring consistent infrastructure across environments.}
        \end{cvitems}
    }

    \cventry
    {Software Engineer}
    {Ford Motor Company}
    {Dearborn, MI}
    {June 2014 - February 2018}
    {
        \begin{cvitems}
            \item{Developed Java-based, multi-threaded application to ingest data into Hadoop with a maximum throughput of 20 gigabits per second. The minimum viable product was delivered 
                  3 months early.}
            \item{Developed a web-based data ingestion monitoring application using Spring Boot and Angular JS. The application was deployed to Pivotal Cloud Foundry (PCF) and provided the 
                  ability to support global ingestion sites from Dearborn.}
            \item{Lead a cross-functional team to improve native Hadoop data ingestion performance from Palo Alto, CA to Dearborn, MI. This project saved the company \$300,000 per year by 
                  eliminating the need to use a WAN accelerator.}
            \item{Implemented a Continuous Integration (CI)/Continuous Deployment (CD) pipeline to allow the application to move from GitHub to the production PCF instance with no manual 
                  intervention required outside of GitHub.}
        \end{cvitems}
    }
    
    \cventry
    {Apprentice Engineer II}
    {DENSO International America, Inc.}
    {Southfield, MI}
    {April 2013 - May 2014}
    {
        \begin{cvitems}
            \item{Modified embedded Telematics Control Unit (TCU) firmware written in C to send commands to the vehicle's CAN bus to demonstrate remote control capability to customers.}
            \item{Developed a vehicle status API in PHP that was used as part of DENSO's 2014 CES demo.}
            \item{Introduced software craftsmanship practices to the team, including code refactoring and reuse.}
        \end{cvitems}
    }
    
    \cventry
    {Engineering Intern}
    {Schweitzer Engineering Laboratories}
    {Plymouth, MI}
    {May 2012 - March 2013}
    {
        \begin{cvitems}
            \item{Programmed communications processors and HMIs (human-machine interfaces).}
            \item{Prepared SCADA points lists (DNP, Modbus, and SEL protocol) and reports for customers.}
            \item{Traveled to the field to install new equipment and troubleshoot previous installations.}
        \end{cvitems}
    }
\end{cventries}

\cvsection{Education}
\begin{cventries}
    \cventry
    {Master of Science in Computer Science}
    {Lawrence Technological University}
    {Southfield, MI}
    {Graduated May 2016}
    {
        \begin{cvitems}
            \item{Developed Internet of Things (IOT) data acquisition platform using Python, Flask, and SQLAlchemy for the Raspberry Pi with a plugin system to allow end users to add custom 
                  sensors.}
            \item{Developed IP phone provisioning system in Python using mutual TLS for secure provisioning.}
        \end{cvitems}
    }
    
    \cventry
    {Bachelor of Science in Computer Science}
    {Lawrence Technological University}
    {Southfield, MI}
    {Graduated May 2014}
    {}
\end{cventries}

% Cancel out the lack of content under the BSCS degree. The template should be updated to fix the need to do this.
\vspace{-\baselineskip}

\lettersection{Personal Projects}

% Cancel out template spacing
\vspace{2.0mm}

\descriptionstyle{
    \begin{cvitems}
        \item{Developed a set of open source Ansible roles to deploy the 2600hz distributed VOIP platform, with an emphasis on support for Google Cloud Platform (GCP). Several enhancements were 
              made based on feedback from several community members.}
    \end{cvitems}
}
\end{document}
