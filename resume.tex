%!TEX TS-program = xelatex
%!TEX encoding = UTF-8 Unicode

% Configuration
\documentclass[11pt, a4paper]{awesome-cv}
\geometry{left=1.4cm, top=.8cm, right=1.4cm, bottom=1.8cm, footskip=.5cm}
\fontdir[fonts/]
\definecolor{awesome}{HTML}{007185}
\setbool{acvSectionColorHighlight}{true}
\renewcommand{\acvHeaderSocialSep}{\quad\textbar\quad}

% Contact Information
\name{Thomas}{Newman}
\position{Software Engineer}
\address{REDACTED}

\mobile{REDACTED}
\email{REDACTED}
\github{tnewman}
\linkedin{thomasjnewman}

\begin{document}

\makecvheader[C]

\cvsection{Skills}

\begin{cvskills}
    \cvskill
    {DevOps}
    {Ansible, AWS (Amazon Web Services), Docker, Gradle, Jenkins, Kubernetes, Linux, Terraform}

    \cvskill
    {Backend}
    {DynamoDB, Flask, Microservices, NodeJS, RDS, REST, SNS, SQS, Spring Boot}

    \cvskill
    {Programming}
    {C\#, Java, JavaScript, Python}
\end{cvskills}

\cvsection{Experience}
\begin{cventries}
    \cventry
    {Senior Software Engineer}
    {Sift, LLC.}
    {Detroit, MI}
    {March 2019-Present}
    {
        \begin{cvitems}
            \item{Architected and implemented APIs in NodeJS based on feedback from designers, front-end developers, and product managers. The APIs used various AWS services (DynamoDB, RDS, S3, SQS, etc.).}
            \item{Architected and implemented next generation cloud infrastructure, including infrastructure as code (Terraform), containers (Docker and Kubernetes), and various AWS services (DynamoDB, RDS, S3, SQS, etc.). This effort significantly reduced site outages and simplified application deployments.}
            \item{Migrated legacy applications and infrastructure, including modifying applications to run in Docker, setting up Docker builds, and setting up CI pipelines.}
        \end{cvitems}
    }

    \cventry
    {Software Engineer}
    {DENSO International America, Inc.}
    {Southfield, MI}
    {February 2018-March 2019}
    {
        \begin{cvitems}
            \item{Architected and wrote microservices in Python (Flask), JavaScript (NodeJS) and Java (Spring Boot) for REST and Pub/Sub-based APIs. The microservices integrated with DynamoDB, SNS, and SQS for a car sharing platform.}
            \item{Wrote a Python library for AWS SNS and SQS. This library allowed all Python microservices to benefit from multi-threaded queue subscriptions, greatly increasing performance.}
            \item{Architected and implemented cloud infrastructure, including infrastructure as code (Terraform), containers (Docker and Kubernetes), and various AWS services (DynamoDB, SNS, and SQS).}
            \item{Implemented Continuous Integration (CI)/Continuous Deployment (CD) pipelines in Jenkins.}
        \end{cvitems}
    }

    \cventry
    {Software Engineer}
    {Ford Motor Company}
    {Dearborn, MI}
    {June 2014 - February 2018}
    {
        \begin{cvitems}
            \item{Developed Java-based, multi-threaded application to ingest data into Hadoop with a maximum throughput of 20 gigabits per second. The minimum viable product was delivered 
                  3 months early.}
            \item{Developed a web-based data ingestion monitoring application using Spring Boot and Angular JS. The application was deployed to Pivotal Cloud Foundry (PCF) and provided the 
                  ability to support global ingestion sites from Dearborn.}
            \item{Lead a cross-functional team to improve native Hadoop data ingestion performance from Palo Alto, CA to Dearborn, MI. This project saved the company \$300,000 per year by 
                  eliminating the need to use a WAN accelerator.}
            \item{Implemented a Continuous Integration (CI)/Continuous Deployment (CD) pipeline to allow the application to move from GitHub to the production PCF instance with no manual 
                  intervention required outside of GitHub.}
        \end{cvitems}
    }
    
    \cventry
    {Apprentice Engineer II}
    {DENSO International America, Inc.}
    {Southfield, MI}
    {April 2013 - May 2014}
    {
        \begin{cvitems}
            \item{Modified embedded Telematics Control Unit (TCU) firmware written in C to send commands to the vehicle's CAN bus to demonstrate remote control capability to customers.}
            \item{Developed a vehicle status API in PHP that was used as part of DENSO's 2014 CES demo.}
            \item{Introduced software craftsmanship practices to the team, including code refactoring and reuse.}
        \end{cvitems}
    }
    
    \cventry
    {Engineering Intern}
    {Schweitzer Engineering Laboratories}
    {Plymouth, MI}
    {May 2012 - March 2013}
    {
        \begin{cvitems}
            \item{Programmed communications processors and HMIs (human-machine interfaces).}
            \item{Prepared SCADA points lists (DNP, Modbus, and SEL protocol) and reports for customers.}
            \item{Traveled to the field to install new equipment and troubleshoot previous installations.}
        \end{cvitems}
    }
\end{cventries}

\cvsection{Education}
\begin{cventries}
    \cventry
    {Master of Science in Computer Science}
    {Lawrence Technological University}
    {Southfield, MI}
    {Graduated May 2016}
    {}

% Cancel out the lack of spacing under the MCSC degree. The template should be updated to fix the need to do this.
\vspace{-\baselineskip}

    \cventry
    {Bachelor of Science in Computer Science}
    {Lawrence Technological University}
    {Southfield, MI}
    {Graduated May 2014}
    {}
\end{cventries}

% Cancel out the lack of content under the BSCS degree. The template should be updated to fix the need to do this.
\vspace{-\baselineskip}

\end{document}
